% Created 2021-12-03 Fri 16:42
% Intended LaTeX compiler: pdflatex
\documentclass[11pt]{article}
\usepackage[utf8]{inputenc}
\usepackage[T1]{fontenc}
\usepackage{graphicx}
\usepackage{grffile}
\usepackage{longtable}
\usepackage{wrapfig}
\usepackage{rotating}
\usepackage[normalem]{ulem}
\usepackage{amsmath}
\usepackage{textcomp}
\usepackage{amssymb}
\usepackage{capt-of}
\usepackage{hyperref}
\author{于承业}
\date{\today}
\title{}
\hypersetup{
 pdfauthor={于承业},
 pdftitle={},
 pdfkeywords={},
 pdfsubject={},
 pdfcreator={Emacs 27.2 (Org mode 9.4.4)}, 
 pdflang={English}}
\begin{document}

\tableofcontents

\section{Lisp}
\label{sec:orgd3d41e7}
\subsection{Problems}
\label{sec:org14930c9}
在实现解释器时(无论是JSON还是Lisp等),一个必须要考虑的问题是多种类型的如何处理、以及嵌套子表达式应该如何表示。
下面谈一谈我在用C++和Rust处理这两个问题时的感慨
\subsubsection{Generic type}
\label{sec:orge336e53}
\label{org7e21ac2}
\begin{verbatim}
using i32 = uint32_t;
using f64 = double;
struct Value{
    value_t type;
    union{i32 Integer; f64 Fp;} data;
    template<class T>
    T get(){
	if(type==Int) return data.Integer;
	return data.Fp;
    }
    Value(f64 _data):type(Double){data.Fp=_data;}
    Value(i32 _data):type(Int){data.Integer=_data;}
    Value():type(Int){data.Integer=0;}
};
// store
auto a = Value(10);
auto b = Value(3.14);

// get
assert(a.get<i32>()==10);
assert(a.get<f64>()==3.14);
\end{verbatim}
以上是我在\href{https://github.com/kabu1204/ILisp}{ILisp}中的实现:使用 \textbf{\textbf{type}} 来记录所存储值的类型;使用 \textbf{\textbf{union}} 存储数据来使用尽量少的空间。这种方式是权衡后的选择。在取值的时候虚要根据type判断get<>()的类型。而且虽然在C++11之后,union可以存储非POD类型的数据了,但是对std::string的支持依然不是很好。导致我只能存储i32和f64两种简单的类型。
相比之下,Rust的枚举类型在应对这种场景时显得更加得心应手:
\label{org135d774}
\begin{verbatim}
#[derive(Debug, Clone)]
pub enum Atomic {
    Number(i32),
    Float(f64),
    Symbol(String)  // TODO: refactor to &str
}

#[derive(Debug, Clone)]
pub enum LispType {
    Atom(Atomic),
    List(Vec<LispType>)
}

pub fn eval(expr: &LispType) {
    match expr {
	LispType::List(subexpr) => {
	    // do something
	    // calculate the value of sub-expression
	},
	LispType::Atom(Atomic::Symbol(symbol_name)) => {
	    // do something
	    // get value from Env
	},
	LispType::Atom(i32_or_f64) => {
	    match i32_or_f64 {
		Atomic::Number(num) => {/* do something */},
		Atomic::Float(fp) => {/* do something */}
	    }
	}
    }
}
\end{verbatim}
以上是我在\href{https://github.com/kabu1204/rlisp}{rlisp}的实现,可以看到,由于
\begin{enumerate}
\item Rust的enum中每种枚举都可以和一个已有类型的值绑定
\item 通过match表达式强大的模式匹配功能,可以 \textbf{\textbf{优雅}} (至少我觉得比较优雅) 地取出所绑定的值。
\end{enumerate}
\subsubsection{Nested sub-expression}
\label{sec:org3ba4d9f}
对比上面的代码(\ref{org7e21ac2} \ref{org135d774}):
\begin{enumerate}
\item 在c++实现中,我没有将类似vector<Value>的结构放在结构体Value的定义中,因为我觉得这使得Value的体积过于膨胀了,一个值类型应该是 \textbf{\textbf{简洁}} 。
因此在c++实现中,我保留了["(", ")"],将其作为边界字符,在eval()函数遇到"("时则进入下一层递归,在遇到")"计算并返回子表达式的值。
\item 在rust实现中,由于enum足够简洁,我直接将子表达式(一个List,和rust的Vec<Lisptype>绑定)也作为一种枚举中的一种类型(实际上List本来就是Lisp的一种类型,这种设计更贴合Lisp)。类似地,eval()函数在match到LispType::List(Vec<LispType>)的时候进入下一层递归,离开作用域时计算并返回子表达式的值即可
\end{enumerate}
\end{document}
